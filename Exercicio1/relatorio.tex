\documentclass[a4paper]{article}

\usepackage[utf8]{inputenc}
\usepackage[portuguese]{babel}
\usepackage[pdftex]{hyperref}
\usepackage{graphicx}
\usepackage{listings}

\begin{document}



\vfill
\begin{center}
    \Huge\bfseries
    Universidade do Minho \\
    \vspace{10mm}
    \includegraphics[width=0.2\textwidth]{logo.jpg}\\[0.5cm]
    \vspace{30mm}
    \LARGE\bfseries
           Prestação de cuidados de saúde \\
    \vspace{30mm}
    \large\bfseries
    Mestrado Integrado em Engenharia Informática \\
    \vspace{20mm}
    Sistemas de Representação de Conhecimento e Raciocínio \\
    (2º Semestre - 17/18)
\end{center}

\vfill

\begin{flushleft}
A77531 \hspace{3mm} Daniel Fernandes Veiga Maia \\
A78034 \hspace{3mm} Diogo Afonso Silva Costa \\
A73909 \hspace{3mm} Francisco Lira Pereira \\
A79607 \hspace{3mm} Marco António Rodrigues Oliveira Silva \\
\end{flushleft}

\vspace{10mm}

\begin{flushright}
Braga \\
2018 Março
\end{flushright}

\thispagestyle{empty}

\newpage

\begin{abstract}

\hspace{3mm} Neste documento será apresentado o trabalho desenvolvido no âmbito da Unidade Curricular de Sistemas de Representação de Conhecimento e Raciocínio do 3º ano do Mestrado Integrado em Engenharia Informática. Foi proposto o desenvolvimento de uma base de conhecimento e respetivos predicados na área da saúde, mais concretamente, com utentes, cuidados e seus respetivos prestadores. Serão explicitados os predicados utilizados para navegar a base de conhecimento bem como os respetivos invariantes, responsáveis por manter a consistência da mesma.


\end{abstract}

\newpage

\tableofcontents

\newpage

\section{Introdução}

\hspace{3mm} O sistema de representação de conhecimento e raciocínio, representado neste projeto, baseia-se na área de prestação de cuidados de saúde. Este tem como objetivo permitir a construção e evolução de uma base de conhecimento que possa retratar as principais componentes no que toca à saúde, nomeadamente, o utente, o prestador do cuidado de saúde, o cuidado propriamente dito e a instituição em que este é realizado.

Para tal, é usada a linguagem de programação em lógica, PROLOG. Esta permite através de um conjunto de predicados, invariantes e estruturas lógicas orquestrar um base de conhecimento consistente e, acima de tudo, evolutiva.

\section{Preliminares}

\hspace{3mm} O PROLOG é uma linguagem de programação em lógica que contém um pequeno conjunto de mecanismos básicos, nomeadamente, unificações, estruturas em árvore e \textit{backtracking} automático. Neste relatório muitos destes mecanismos encontram-se subjacentes e são deveras importantes para a compreensão do mesmo. \cite{Bratko:1990:PPA:533072}

Além disso, são explicados de forma extensiva os diferentes predicados que compõem a base de conhecimento. Estes seguem o clausulado de \textit{Horn}, que diz que apenas existe um termo positivo (conclusão) e que todos os predicados estão quantificados universalmente. Efetivamente, estes são fórmulas fechadas visto que todas as variáveis têm significado, não importa o seu valor.

Por fim, importa notar que o algoritmo de resolução adotado pelo PROLOG segue o conceito \textit{modus ponens}, que permite derivar como verdadeiro uma conclusão de uma cláusula, e o \textit{modus tollens}, que permite dirigir a prova para um ponto em particular. É a conjunção de este e outros mecanismos que permite que o PROLOG para ser usado para a representação de conhecimento e raciocínio e, consequentemente, neste projeto.

\section{Descrição do Trabalho e Análise de Resultados}

\subsection{Predicados base}
\hspace{3mm} A área da prestação de cuidados de saúde é caracterizada por um sistema de representação de conhecimento e raciocínio composto por quatro predicados fundamentais.

\begin{itemize}
    \item \textbf{utente}: \#IdUt, Nome, Idade, Morada \( \rightarrow \) \{ V,F \} 
    \item \textbf{prestador}: \#IdPrest, Nome, Especialidade, \#IdInst \( \rightarrow \) \{ V,F \}
    \item \textbf{cuidado}: Data, \#IdUt, \#IdPrest, Descrição, Custo \( \rightarrow \) \{ V,F \}
    \item \textbf{instituicao}: \#IdInst, Nome, Cidade \( \rightarrow \) \{ V,F \}
\end{itemize}

O último predicado surgiu da necessidade de identificar os cuidados de saúde prestados numa determinada cidade. No entanto, o único conhecimento disponível era a morada do cliente. Não foi do agrado da equipa utilizar esta como forma de unificar o conhecimento para concluir quais os cuidados de saúde prestados numa determinada cidade, visto que os utentes poderiam realizar consultas fora do seu local de residência. Assim sendo, foi estendida a informação referente à instituição de forma a obter uma base de conhecimento mais robusta.

Assim sendo, são estas as ferramentas iniciais que possibilitam a criação de uma base de conhecimento ligada à prestação de cuidados de saúde, com possibilidade de evolução do próprio conhecimento.


\subsection{Extensões dos predicados}

\subsubsection{Registar utentes, prestadores e cuidados de saúde}

\hspace{3mm} Após a declaração dos predicados \textbf{utente}, \textbf{prestador}, \textbf{cuidado} e \textbf{instituicao}, surge a necessidade de permitir a que a base de conhecimento possa evoluir. Para tal, foi necessária a implementação do predicado \textbf{registar}. No entanto, a inserção de conhecimento pode levar a um estado inconsistente. A título exemplificativo, tomando a hipótese do \texttt{utente(1, andre, 25, lisboa).} já se encontrar a base de conhecimento, a inserção do \texttt{utente(1, guilherme, 60, faro).} levaria a que houvesse dois utentes com o mesmo identificador (\textit{IdUt}), o que origina uma inconsistência na base de conhecimento. A solução surge na forma de invariantes. Estes representam testes à consistência da base de conhecimento, isto é, permite que haja as condições adequadas para que o sistema possa crescer.

Deste modo, foram adicionadas invariantes que não permitem que se possa inserir dois utentes com o mesmo \textit{id}. O mesmo foi aplicado ao \textit{id} dos prestadores e das instituições. Além disso, o predicado \textbf{cuidado} não poderá aparecer em duplicado na base de conhecimento, nem referenciar utentes e prestadores que ainda não existam, obrigando desta forma a que estes sejam adicionados primeiro.

Assim sendo, torna-se possível que o sistema evolua de forma consistente através do predicado \textbf{registar}. No entanto, é necessário que este realize os devidos testes antes de dar a inserção como concluída. Para tal foi construído o predicado \textbf{solucoes}.

\begin{lstlisting}[xleftmargin=.2\textwidth]
solucoes(X,Y,Z) :- findall(X,Y,Z).
\end{lstlisting}

Este coloca na lista Z o resultado, na forma de X, de todas as unificações do predicado Y com a base de conhecimento. De seguida, insere-se o novo termo através do predicado \texttt{assert}. Por fim, são testados todos os invariantes, guardados em Z, usando o predicado \texttt{teste}. Caso algum destes falhe, o termo inserido é retirado da base de conhecimento através do predicado \texttt{retract}. 

\vspace{3mm}

\begin{lstlisting}[xleftmargin=.2\textwidth]
insere(P) :- assert(P).
insere(P) :- retract(P), !, fail.

registar( Termo ) :- 
    solucoes(Inv, +Termo :: Inv, S),
    insere(Termo),
    teste(S).
\end{lstlisting}

\vspace{3mm}

Todos estes componentes são necessários para garantir, através do predicado \texttt{registar}, que a base de conhecimento se mantenha consistente ao longo do seu crescimento.


\subsubsection{Remover utentes, prestadores e cuidados de saúde}

\hspace{3mm} De facto, assim como é possível evoluir a base de conhecimento inserindo conhecimento também o é removendo conhecimento. Para tal, é o usado o predicado \texttt{remover}. Este, à semelhança do da \texttt{registar} permite que a base de conhecimento cresça mas desta vez com a remoção de predicados. No entanto, esta deve ser controlada de forma a não causar inconsistências. Para tal, implementou-se alguns invariantes, nomeadamente, a impossibilidade de remover utentes e prestadores que ainda tenham a si associados cuidados de saúde.

Desta forma, tem-se todos os componentes necessários para a implementação do predicado \texttt{remover}. Primeiramente, são recolhidos numa lista, todos os invariantes que se referem a remoção de conhecimento. De seguida, é removido o termo indicado. Por fim, é realizado um teste a todos os invariantes. Caso algum estes não tenha valor de verdade positivo, então é colocado novamente na base de conhecimento.

\vspace{3mm}

\begin{lstlisting}[xleftmargin=.2\textwidth]
remove(P) :- retract(P).
remove(P) :- assert(P), !, fail.

remover( Termo ) :- 
    solucoes(Inv, -Termo :: Inv, S),
    remove(Termo),
    teste(S).
\end{lstlisting}

\vspace{3mm}

Assim sendo, com os predicados \texttt{registar} e \texttt{remover} é possível crescer a base de conhecimento de forma \textbf{consistente}.

\subsubsection{Identificar utentes por critérios de seleção}

\hspace{3mm} Este predicado, denominado \textit{consultaUtente}, permite a procura de utentes através de cada um dos seus átomos. Por outras palavras, é possível procurar utentes pelo seu identificador, nome, idade ou morada. Permite-se também a procura através do uso simultâneo de uma ou mais das variáveis mencionadas. É de notar que efetuar tal procura através do identificador em conjunto com qualquer argumento será redundante, visto que, graças ao invariante por referência de \textit{utente}, qualquer utente que exista na base de conhecimento terá um ID único e específico a si.

\par Para tal, recorre-se ao predicado \textit{solucoes}, com o qual se gera uma lista contendo toda a informação dos \textit{utentes} cujos argumentos introduzidos coincidem com a respetiva informação. Este predicado receberá como argumentos a estrutura de cada elemento da lista, os predicados que usará para testar se contêm a informação procurada e a variável na qual será escrita a lista. Neste caso, os elementos da lista seguirão a estrutura de \textit{utente} e o predicado de teste será o próprio predicado \textit{utente}. Com o intuito de tomar partido disto, será necessário adicionar mais um argumento ao \textit{consultaUtente}, no qual a lista será guardada.

\subsubsection{Identificar as instituições prestadoras de cuidados de saúde}

\hspace{3mm} O predicado \textit{consultaInstituicoes} permite visualizar uma lista de todas as \textit{instituicoes} presentes na base de conhecimento cujos prestadores tenham efetuado cuidados médicos. De modo semelhante ao predicado anterior, é possível efetuar a procura através do identificador, nome e/ou cidade correspondente à instituição.
\par Como no predicado anterior, recorrer-se-á ao predicado \textit{solucoes} e, como tal, necessitar-se-á de uma variável adicional no predicado \textit{consultaInstituicoes}. Neste caso, os elementos da lista seguirão a estrutura de \textit{instituicao} e os predicados de teste serão o predicado \textit{instituicao}, bem como os predicados \textit{prestador} e \textit{cuidado}. Isto deve-se ao facto de que se pretende que apenas as instituições que dispõem de cuidados médicos sejam consideradas neste predicado.
\par Como se recorre a vários predicados para a procura, existe a possibilidade de ser introduzida na lista mais do que uma instância de uma dada instituição. Como tal, recorreu-se ao predicado \textit{sort} para eliminar duplicados da lista.

\subsubsection{Identificar cuidados de saúde prestados}

\hspace{3mm} 
Com o predicado \textit{consultaCuidados}, é possível identificar os cuidados de saúde prestados numa determinada instituição, cidade ou até mesmo numa determinada data.
\par A construção deste foi baseada no encadeamento de outros predicados mais simples. Assim, conclui-se que os argumentos necessários serão as variáveis \emph{IDI}, \emph{Ci}, \emph{D} e finalmente \emph{S}. No que diz respeito à variável \emph{IDI}, esta representa o identificador da instituição da qual se pretende fazer a pesquisa na base de conhecimento. De forma análoga, \emph{Ci} representa a cidade onde a instituição se encontra sediada na qual o cuidado foi prestado e finalmente \emph{D} representa a data sobre a qual a pesquisa deverá incidir. A variável \emph{S} é utilizada na apresentação do resultado do predicado.
\par Mais uma vez, a utilização do predicado soluções foi determinante para a construção da solução. Deste modo, o primeiro elemento que o \emph{solucoes} receberá será o formado dos elementos constituintes da lista, seguido pelos predicados \textit{instituicao}, \textit{prestador} e finalmente \textit{cuidado}. De salientar que cada um dos predicados referidos anteriormente recebe as variáveis passadas como argumento, tendo especial atenção neste caso às variáveis \emph{IDI} e \emph{IDP} que permitem estabelecer a ligação entre os cuidados e os prestadores que por sua vez prestaram um determinado serviço numa instituição.

\subsubsection{Identificação de utentes}

\hspace{3mm} Este predicado é uma extensão do predicado \textit{consultaUtente} e possibilita a procura de todos os utentes que recorreram a um determinado prestador, instituição e/ou especialidade médica. Atinge isto procurando \textit{prestadores} cujo identificador, especialidade e/ou instituição médica coincidem com o(s) procurado(s) e, tendo encontrado \textit{prestadores} válidos, identifica os utentes que recorreram a estes serviços procurando \textit{cuidados} cujo identificador corresponda aos \textit{prestadores} em questão.
\par Para tal, utiliza-se o predicado \textit{solucoes} de novo, desta vez com os elementos da lista seguindo a estrutura de \textit{utente} e com os predicados de teste \textit{utente}, \textit{prestador} e \textit{cuidado}. Fazer-se-á uso do predicado \textit{sort} mais uma vez, de modo a evitar a existência de informação repetida na lista resultante.


\subsubsection{Instituições/prestadores a que um utente já recorreu}

\hspace{3mm} O predicado \textit{todasInstPrest} permite identificar quais prestadores e/ou instituições nos quais um determinado utente recebeu cuidados médicos. Deste modo, recebe como argumento o identificador de um dado utente e uma variável na qual se colocará a lista de pares de identificadores (prestador, instituição).
\par Para tal, recorrer-se-á ao predicado \textit{solucoes}, cujos argumentos serão a estrutura de cada elemento da lista, os predicados \textit{prestador} e \textit{cuidado} e a variável responsável por demonstrar a lista. Cada elemento da lista é um tuplo constituído pelo identificador do prestador e da instituição, respetivamente.

\subsubsection{Calculo do custo total dos cuidados de saúde}

\hspace{3mm} O predicado \textbf{totalCuidados} determina a soma de todos os custos que se encontram descritos na base de conhecimento com base nos argumentos definidos. Este predicado aceita o \emph{ID} de um \textit{utente}, uma \textit{especialidade}, o \emph{ID} de um prestador, uma \textit{data} ou qualquer combinação possível destas informações. Sendo assim, é construída uma lista que alberga todos os custos encontrados na base de conhecimento. Finalmente, recorreu-se ao predicado \textbf{somaL} que recebe uma lista e apresenta a soma de todos os seus elementos.

\subsection{Funcionalidades acrescentadas}

\subsubsection{Receitas da instituição}

\hspace{3mm} Uma vez que nos encontramos numa área em que é tratada informação com custos associados, achou-se interessante construir um predicado \textit{receitasInst} que, apenas com base na instituição, apresente o total dos custos associados à mesma.
\par Deste modo, para o desenvolvimento deste predicado, foi utilizado o predicado soluções. Para a obtenção da informação correta, é necessário combinar as informações presentes na base de conhecimento que estão associadas a \emph{cuidado} e \emph{prestador}. Assim, no segundo parâmetro do predicado soluções, é explicitado que o \emph{ID} de um prestador em que a instituição seja a que foi fornecida, seja o mesmo presente na base de conhecimento associado ao cuidado, uma vez que a ligação entre instituições e cuidados é estabelecida através dos prestadores de cuidados. Deste modo, é construida mais uma vez uma lista com todos os custos encontrados na base de conhecimento e finalmente é utilizado o predicado \textbf{somaL} para que a soma de todos os valores da lista sejam calculada.

\subsubsection{Lista dos cuidados de saúde de um utente}

\hspace{3mm} O predicado \texttt{cuidadosUtente} tem como propósito apresentar um lista com todos os cuidados de um determinado utente. Esta encontrar-se-á ordenada de forma decrescente em relação ao número de ocorrências de um determinado cuidado. De forma a atingir este objetivo são necessários quatro predicados: \texttt{solucoes}, \texttt{pairFreq}, \texttt{sort} e \texttt{reverse}.

Inicialmente, unificam-se todos os predicados com um determinado \textit{id} de utente. O resultado dessa unificação é colocado numa lista \textbf{S}. Conjuntamente, é testado o predicado \texttt{pairFreq} que associa a cada especificação de um cuidado o número de vezes que esta ocorre, com a ajuda do predicado \texttt{conta}. Por fim, é feita a conjunção com os predicados \texttt{sort} e \texttt{reverse}, que ordenam e invertem a ordem dos elementos da lista, respetivamente.

Assim sendo, é possível perceber quais os cuidados mais frequentes para um determinado utente.

\subsubsection{Número de prestadores por instituição}

\hspace{3mm} O predicado \textit{quantosPrest} permite determinar o número de prestadores de cuidados médicos trabalham numa determinada instituição. Como tal, recebe apenas o identificador único da instituição cuja contagem se quer efetuar. O predicado atinge isto gerando uma lista de prestadores cujo identificador da instituição para qual trabalha corresponde ao identificador da instituição em questão e calcula o comprimento da lista resultante. Para tal, recorre-se ao predicado \textit{solucoes} para gerar a lista e ao predicado \textit{comprimento} para calcular o número de elementos na mesma. 
\par No caso de \textit{solucoes}, são passados como argumentos a estrutura dos elementos da lista e o predicado \textit{prestador}. Por sua vez, o predicado \textit{comprimento} receberá a lista gerada e determinará o seu comprimento.

\subsubsection{Cuidados oferecidos por uma instituição}

\hspace{3mm} Achou-se também interessante oferecer a um possível utilizador um catálogo de cuidados que já foram prestados por uma determinada instituição com o predicado \emph{cuidadosInst}. Este catálogo apresenta a descrição do mesmo, uma vez que esta última secção pode conter informações relevantes para o utilizador.
\par Mais uma vez, recorreu-se ao predicado soluções como forma de construir a lista a apresentar com a informação pedida.
\par Deste modo, apenas se necessita de fornecer o \emph{ID} da instituição em causa. Uma vez que, cada prestador tem associada a si uma instituição pelo seu número de identificação, utilizar-se-á o predicado prestador para estabelecer a ligação com os cuidados prestados, verificando todos os prestadores que têm associados a si o número identificativo da instituição fornecida ao predicado. Assim apenas serão apresentados resultados referentes à instituição previamente definida.

\subsubsection{Relatório mensal de uma instituição}

\hspace{3mm} Com o predicado \emph{relatContas} é possível obter uma visão geral de todos os cuidados prestados sendo possível fornecer um mês, ano, e finalmente um identificador de uma instituição. 

Para a construção deste, foram utilizados dois predicados auxiliares. Inicialmente, é construida uma lista com todos os cuidados que correspondem aos argumentos fornecidos. De seguida é utilizado o predicado \emph{calcTotal}, que irá calcular o total de despesas presentes na lista resultante do predicado \emph{solucoes}. Por último, com o predicado \emph{addTotal} é construída a lista final a apresentar adicionando apenas no final da lista resultante do predicado \emph{solucoes} um elemento que especifica o total das despesas para uma melhor visão do panorama mensal.

\newpage
\section{Conclusões e Sugestões}

\hspace{3mm} Este projeto permite a realização de questões a uma base de conhecimento relacionada com a prestação de cuidados de saúde.

Efetivamente, a base de conhecimento concebida permite responder a uma panóplia de questões, no mínimo interessantes. A título exemplificativo, é possível saber qual quais os cuidados que um utente recebeu, quais os prestadores de cuidados numa determinada instituição ou mesmo qual o relatório de cuidados de saúde numa instituição em específico.

No inicio tivemos algumas dificuldades em perceber ao certo o que teríamos de fazer e como o deveríamos de implementar para conseguir respeitar todos os requisitos, mas com algum esforço, facilmente conseguiu-se superar estas dificuldades e desenvolver predicados que satisfizessem as questões predefinidas. Desta forma, tem-se como exemplo a utilização do \textit{sort}. Este permitiu que se resolvesse o problema da existência de parâmetros duplicados em listas. É um facto que este também altera a ordem da ocorrência destes elementos na lista, mas como esta não é importante, acaba por ser até uma ajuda no momento em que, enquanto utilizadores, é feita a procura, a lista tem variados elementos e queremos encontrar entre eles um especial, basta ajustar o nosso critério de procura pois se sabe como ela está organizada.

No entanto, a área da prestação de cuidados de saúde é bastante abrangente e a base de conhecimento aqui desenvolvida pode, e deve, ser evoluída em iterações futuras do projeto. Desta forma, propõe-se o acrescento de um predicado "receitas médicas" que será associado aos utentes, com os vários medicamentos prescritos. A adição deste novo predicado permitiria formular um novo conjunto de questões bastantes pertinentes, como por exemplo, que medicamentos se encontra a tomar um determinado utente ou quais os medicamentos mais receitados por instituição num determinado ano. Assim sendo, é possível constatar que esta é um base de conhecimento com bastantes frentes de crescimento.

Assim, a realização deste primeiro trabalho foi bastante importante para entender como funciona uma linguagem de programação em lógica e os problemas que esta se propõe a resolver. Em especial, como funciona e como se resolve este tipo de problemas em PROLOG. Conseguiu-se consolidar os conhecimentos adquiridos nas aulas práticas, e também complementar estes com alguma pesquisa realizada ao longo do projeto. Conclui-se então com um sentimento de satisfação derivado do trabalho desenvolvido, sendo que esta é uma boa maneira de introduzir a matéria lecionada na UC.

\newpage
\bibliography{mybib}{}
\bibliographystyle{acm}

\end{document}

